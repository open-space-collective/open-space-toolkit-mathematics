Structure

\href{https://github.com/open-space-collective/open-space-toolkit-mathematics/actions/workflows/build-test.yml}{\tt } \href{https://codecov.io/gh/open-space-collective/open-space-toolkit-mathematics}{\tt } \href{https://open-space-collective.github.io/open-space-toolkit-mathematics}{\tt } \href{https://badge.fury.io/gh/open-space-collective%2Fopen-space-toolkit-mathematics}{\tt } \href{https://badge.fury.io/py/open-space-toolkit-mathematics}{\tt } \href{https://opensource.org/licenses/Apache-2.0}{\tt }

Geometry, curve fitting, optimization.

\subsection*{Getting Started}

Want to get started? This is the simplest and quickest way\+:

\href{https://mybinder.org/v2/gh/open-space-collective/open-space-toolkit/master?urlpath=lab/tree/notebooks}{\tt }

{\itshape Nothing to download or install! This will automatically start a \href{https://jupyterlab.readthedocs.io/en/stable/}{\tt Jupyter\+Lab} environment in your browser with Open Space Toolkit libraries and example notebooks ready to use.}

\subsubsection*{Alternatives}

\paragraph*{Docker Images}

\href{https://www.docker.com/}{\tt Docker} must be installed on your system.

\subparagraph*{i\+Python}

The following command will start an \href{https://ipython.org/}{\tt i\+Python} shell within a container where the O\+S\+Tk components are already installed\+:


\begin{DoxyCode}
docker run -it openspacecollective/open-space-toolkit-mathematics-python
\end{DoxyCode}


Once the shell is up and running, playing with it is easy\+:


\begin{DoxyCode}
\textcolor{keyword}{import} numpy
\textcolor{keyword}{from} ostk.mathematics.geometry \textcolor{keyword}{import} Angle
\textcolor{keyword}{from} ostk.mathematics.geometry.d3.transformations.rotations \textcolor{keyword}{import} Quaternion
\textcolor{keyword}{from} ostk.mathematics.geometry.d3.transformations.rotations \textcolor{keyword}{import} RotationVector

rv = RotationVector(numpy.array([[0.0], [0.0], [1.0]], dtype=float), Angle.degrees(15.0)) \textcolor{comment}{# Construct
       rotation vector}

q\_AB = Quaternion(1.0, 2.0, 3.0, 4.0, Quaternion.Format.XYZS).to\_normalized() \textcolor{comment}{# Construct quaternion and
       normalize}
q\_BC = Quaternion.rotation\_vector(rv) \textcolor{comment}{# Construct quaternion from rotation vector}

q\_AC = q\_AB * q\_BC \textcolor{comment}{# Multiply quaternions}
\end{DoxyCode}


{\itshape Tip\+: Use tab for auto-\/completion!}

\subparagraph*{Jupyter\+Lab}

The following command will start a \href{https://jupyterlab.readthedocs.io/en/stable/}{\tt Jupyter\+Lab} server within a container where the O\+S\+Tk components are already installed\+:


\begin{DoxyCode}
docker run --publish=8888:8888 openspacecollective/open-space-toolkit-mathematics-jupyter
\end{DoxyCode}


Once the container is running, access \href{http://localhost:8888/lab}{\tt http\+://localhost\+:8888/lab} and create a Python 3 Notebook.

\subsection*{Installation}

\subsubsection*{C++}

The binary packages are hosted using \href{https://github.com/open-space-collective/open-space-toolkit-mathematics/releases}{\tt Git\+Hub Releases}\+:


\begin{DoxyItemize}
\item Runtime libraries\+: {\ttfamily open-\/space-\/toolkit-\/mathematics-\/\+X.\+Y.\+Z-\/1.\+x86\+\_\+64-\/runtime}
\item C++ headers\+: {\ttfamily open-\/space-\/toolkit-\/mathematics-\/\+X.\+Y.\+Z-\/1.\+x86\+\_\+64-\/devel}
\item Python bindings\+: {\ttfamily open-\/space-\/toolkit-\/mathematics-\/\+X.\+Y.\+Z-\/1.\+x86\+\_\+64-\/python}
\end{DoxyItemize}

\paragraph*{Debian / Ubuntu}

After downloading the relevant {\ttfamily .deb} binary packages, install\+:


\begin{DoxyCode}
apt install open-space-toolkit-mathematics-*.deb
\end{DoxyCode}


\paragraph*{Fedora / Cent\+OS}

After downloading the relevant {\ttfamily .rpm} binary packages, install\+:


\begin{DoxyCode}
dnf install open-space-toolkit-mathematics-*.rpm
\end{DoxyCode}


\subsubsection*{Python}

Install from \href{https://pypi.org/project/open-space-toolkit-mathematics/}{\tt Py\+PI}\+:


\begin{DoxyCode}
pip install open-space-toolkit-mathematics
\end{DoxyCode}


\subsection*{Documentation}

Documentation is available here\+:


\begin{DoxyItemize}
\item \href{https://open-space-collective.github.io/open-space-toolkit-mathematics}{\tt C++}
\item \href{./bindings/python/docs}{\tt Python}
\end{DoxyItemize}

$<$details$>$

The library exhibits the following structure\+:


\begin{DoxyCode}
├── Objects
│   ├── Vector
│   ├── Matrix
│   └── Interval
├── Geometry
│   ├── 2D
|   │   ├── Objects
│   │   │   ├── Point
│   │   │   ├── Point Set
│   │   │   ├── Line
│   │   │   ├── Line String
│   │   │   ├── Multi Line String
│   │   │   └── Polygon
│   │   ├── Intersection
│   │   └── Transformations
│   │       ├── Identity
│   │       ├── Translation
│   │       ├── Rotation
│   │       ├── Reflection
│   │       ├── Scaling
│   │       └── Shear
│   ├── 3D
|   │   ├── Objects
│   │   │   ├── Point
│   │   │   ├── Point Set
│   │   │   ├── Line
│   │   │   ├── Ray
│   │   │   ├── Segment
│   │   │   ├── Line String
│   │   │   ├── Polygon
│   │   │   ├── Plane
│   │   │   ├── Cuboid
│   │   │   ├── Sphere
│   │   │   ├── Ellipsoid
│   │   │   ├── Cone
│   │   │   ├── Pyramid
│   │   │   └── Composite
│   │   ├── Intersection
│   │   └── Transformations
│   │       ├── Identity
│   │       ├── Translation
│   │       ├── Rotations
│   │       │   ├── Quaternion
│   │       │   ├── Euler Angle
│   │       │   ├── Rotation Vector
│   │       │   └── Rotation Matrix
│   │       ├── Reflection
│   │       ├── Scaling
│   │       └── Shear
├── Dynamics
│   ├── State
│   ├── Solvers
│   │   ├── Runge–Kutta 4 (RK4)
│   │   ├── Dormand–Prince 5 (DP5)
│   │   └── Runge–Kutta–Fehlberg 78 (F78)
│   └── Systems
├── Curve Fitting
│   ├── Interpolation
│   │   ├── Linear
│   │   ├── Cubic Spline
│   │   └── Lagrange
│   └── Smoothing
├── Optimization
│   ├── Problem
│   └── Algorithms
│       ├── Gradient Descent
│       └── Evolutionary
│           ├── Genetic
│           ├── Differential Evolution
│           └── Swarm
└── Statistics
\end{DoxyCode}


$<$/details$>$

\subsection*{Tutorials}

Tutorials are available here\+:


\begin{DoxyItemize}
\item \href{./tutorials/cpp}{\tt C++}
\item \href{./tutorials/python}{\tt Python}
\end{DoxyItemize}

\subsection*{Features}

\subsubsection*{Geometry Queries}


\begin{DoxyItemize}
\item {\ttfamily ○}\+: query only
\item {\ttfamily ✔}\+: query / intersection set
\item \+: to be implemented
\item {\ttfamily -\/}\+: undefined
\end{DoxyItemize}

\paragraph*{3D}

\tabulinesep=1mm
\begin{longtabu} spread 0pt [c]{*{15}{|X[-1]}|}
\hline
\rowcolor{\tableheadbgcolor}\textbf{ Intersect }&\textbf{ Point }&\textbf{ Point Set }&\textbf{ Line }&\textbf{ Ray }&\textbf{ Segment }&\textbf{ Line String }&\textbf{ Polygon }&\textbf{ Plane }&\textbf{ Cuboid }&\textbf{ Sphere }&\textbf{ Ellipsoid }&\textbf{ Cone }&\textbf{ Pyramid }&\textbf{ Composite  }\\\cline{1-15}
\endfirsthead
\hline
\endfoot
\hline
\rowcolor{\tableheadbgcolor}\textbf{ Intersect }&\textbf{ Point }&\textbf{ Point Set }&\textbf{ Line }&\textbf{ Ray }&\textbf{ Segment }&\textbf{ Line String }&\textbf{ Polygon }&\textbf{ Plane }&\textbf{ Cuboid }&\textbf{ Sphere }&\textbf{ Ellipsoid }&\textbf{ Cone }&\textbf{ Pyramid }&\textbf{ Composite  }\\\cline{1-15}
\endhead
{\bfseries Point} &&&○ &○ &&&&✔ &○ &○ &○ &&&\\\cline{1-15}
{\bfseries Point Set} &&&&&&&&✔ &○ &○ &○ &&&\\\cline{1-15}
{\bfseries Line} &○ &&&&&&&✔ &○ &○ &✔ &&&\\\cline{1-15}
{\bfseries Ray} &○ &&&&&&&✔ &&○ &✔ &&&\\\cline{1-15}
{\bfseries Segment} &&&&&&&&✔ &&○ &✔ &&&\\\cline{1-15}
{\bfseries Line String} &&&&&&&&&&&&&&\\\cline{1-15}
{\bfseries Polygon} &&&&&&&&&&&&&&\\\cline{1-15}
{\bfseries Plane} &✔ &✔ &✔ &✔ &✔ &&&&&○ &○ &&&\\\cline{1-15}
{\bfseries Cuboid} &○ &○ &○ &&&&&&&&&&&\\\cline{1-15}
{\bfseries Sphere} &○ &○ &○ &○ &○ &&&○ &&&&&○ &\\\cline{1-15}
{\bfseries Ellipsoid} &○ &○ &✔ &✔ &✔ &&&○ &&&&✔ &✔ &\\\cline{1-15}
{\bfseries Cone} &&&&&&&&&&&✔ &&&\\\cline{1-15}
{\bfseries Pyramid} &&&&&&&&&&○ &✔ &&&\\\cline{1-15}
{\bfseries Composite} &&&&&&&&&&&&&&\\\cline{1-15}
\end{longtabu}
\tabulinesep=1mm
\begin{longtabu} spread 0pt [c]{*{15}{|X[-1]}|}
\hline
\rowcolor{\tableheadbgcolor}\textbf{ Contain }&\textbf{ Point }&\textbf{ Point Set }&\textbf{ Line }&\textbf{ Ray }&\textbf{ Segment }&\textbf{ Line String }&\textbf{ Polygon }&\textbf{ Plane }&\textbf{ Cuboid }&\textbf{ Sphere }&\textbf{ Ellipsoid }&\textbf{ Cone }&\textbf{ Pyramid }&\textbf{ Composite  }\\\cline{1-15}
\endfirsthead
\hline
\endfoot
\hline
\rowcolor{\tableheadbgcolor}\textbf{ Contain }&\textbf{ Point }&\textbf{ Point Set }&\textbf{ Line }&\textbf{ Ray }&\textbf{ Segment }&\textbf{ Line String }&\textbf{ Polygon }&\textbf{ Plane }&\textbf{ Cuboid }&\textbf{ Sphere }&\textbf{ Ellipsoid }&\textbf{ Cone }&\textbf{ Pyramid }&\textbf{ Composite  }\\\cline{1-15}
\endhead
{\bfseries Point} &&&-\/ &-\/ &-\/ &-\/ &-\/ &-\/ &-\/ &-\/ &-\/ &-\/ &-\/ &\\\cline{1-15}
{\bfseries Point Set} &&&-\/ &-\/ &-\/ &-\/ &-\/ &-\/ &-\/ &-\/ &-\/ &-\/ &-\/ &\\\cline{1-15}
{\bfseries Line} &✔ &&&&&&-\/ &-\/ &-\/ &-\/ &-\/ &-\/ &-\/ &\\\cline{1-15}
{\bfseries Ray} &✔ &✔ &-\/ &&&&-\/ &-\/ &-\/ &-\/ &-\/ &-\/ &-\/ &\\\cline{1-15}
{\bfseries Segment} &✔ &&-\/ &-\/ &&&-\/ &-\/ &-\/ &-\/ &-\/ &-\/ &-\/ &\\\cline{1-15}
{\bfseries Line String} &&&-\/ &-\/ &&&-\/ &-\/ &-\/ &-\/ &-\/ &-\/ &-\/ &\\\cline{1-15}
{\bfseries Polygon} &&&-\/ &-\/ &&&&-\/ &-\/ &-\/ &-\/ &-\/ &-\/ &\\\cline{1-15}
{\bfseries Plane} &✔ &✔ &✔ &✔ &✔ &&&&-\/ &-\/ &-\/ &-\/ &-\/ &\\\cline{1-15}
{\bfseries Cuboid} &✔ &✔ &-\/ &-\/ &&&&-\/ &&&&&&\\\cline{1-15}
{\bfseries Sphere} &✔ &✔ &-\/ &-\/ &-\/ &-\/ &-\/ &-\/ &&&&&&\\\cline{1-15}
{\bfseries Ellipsoid} &✔ &✔ &-\/ &-\/ &-\/ &-\/ &-\/ &-\/ &&&&&&\\\cline{1-15}
{\bfseries Cone} &&&&&&-\/ &-\/ &-\/ &&&&&&\\\cline{1-15}
{\bfseries Pyramid} &✔ &&&&&&&-\/ &&&&&&\\\cline{1-15}
{\bfseries Composite} &&&&&&&&&&&&&&\\\cline{1-15}
\end{longtabu}
\subsection*{Setup}

\subsubsection*{Development Environment}

Using \href{https://www.docker.com}{\tt Docker} for development is recommended, to simplify the installation of the necessary build tools and dependencies. Instructions on how to install Docker are available \href{https://docs.docker.com/install/}{\tt here}.

To start the development environment\+:


\begin{DoxyCode}
make start-development
\end{DoxyCode}


This will\+:


\begin{DoxyEnumerate}
\item Build the {\ttfamily openspacecollective/open-\/space-\/toolkit-\/mathematics-\/development} Docker image.
\item Create a development environment container with local source files and helper scripts mounted.
\item Start a {\ttfamily bash} shell from the {\ttfamily ./build} working directory.
\end{DoxyEnumerate}

If installing Docker is not an option, you can manually install the development tools (G\+CC, C\+Make) and all required dependencies, by following a procedure similar to the one described in the \href{./docker/development/Dockerfile}{\tt Development Dockerfile}.

\subsubsection*{Build}

From the {\ttfamily ./build} directory\+:


\begin{DoxyCode}
cmake ..
make
\end{DoxyCode}


{\itshape Tip\+: {\ttfamily helpers/build.\+sh} simplifies building from within the development environment.}

\subsubsection*{Test}

To start a container to build and run the tests\+:


\begin{DoxyCode}
make test
\end{DoxyCode}


Or to run them manually\+:


\begin{DoxyCode}
./bin/open-space-toolkit-mathematics.test
\end{DoxyCode}


{\itshape Tip\+: {\ttfamily helpers/test.\+sh} simplifies running tests from within the development environment.}

\subsection*{Dependencies}

\tabulinesep=1mm
\begin{longtabu} spread 0pt [c]{*{4}{|X[-1]}|}
\hline
\rowcolor{\tableheadbgcolor}\textbf{ Name }&\textbf{ Version }&\textbf{ License }&\textbf{ Link  }\\\cline{1-4}
\endfirsthead
\hline
\endfoot
\hline
\rowcolor{\tableheadbgcolor}\textbf{ Name }&\textbf{ Version }&\textbf{ License }&\textbf{ Link  }\\\cline{1-4}
\endhead
Pybind11 &2.\+6.\+1 &B\+S\+D-\/3-\/\+Clause &\href{https://github.com/pybind/pybind11}{\tt github.\+com/pybind/pybind11} \\\cline{1-4}
Eigen &3.\+3.\+7 &M\+P\+L2 &\href{http://eigen.tuxfamily.org/index.php}{\tt eigen.\+tuxfamily.\+org} \\\cline{1-4}
Geometric Tools Engine &3.\+28 &Boost Software License &\href{https://www.geometrictools.com}{\tt geometrictools.\+com} \\\cline{1-4}
Core &master &Apache License 2.\+0 &\href{https://github.com/open-space-collective/open-space-toolkit-core}{\tt github.\+com/open-\/space-\/collective/open-\/space-\/toolkit-\/core} \\\cline{1-4}
\end{longtabu}
\subsection*{Contribution}

Contributions are more than welcome!

Please read our \hyperlink{_c_o_n_t_r_i_b_u_t_i_n_g_8md}{contributing guide} to learn about our development process, how to propose fixes and improvements, and how to build and test the code.

\subsection*{Special Thanks}

\href{https://www.loftorbital.com/}{\tt }

\subsection*{License}

Apache License 2.\+0 